\chapter{Literature Survey}
\label{chap3}

There are several instructional operating systems available like Nachos, Topsy etc. developed by various universities.\\

Not Another Completely Heuristic Operating System, or Nachos, is one such instructional software developed at the University of California, Berkeley. Nachos supports threads, user-level processes, virtual memory and interrupt-driven input output devices. The only difference between Nachos and a real operating system is that Nachos runs as a single Unix process. It also supports synchronization and networking.\\

Low-level Nachos routines frequently disable and re-enable interrupts to achieve mutual exclusion. Synchronization facilities are provided through semaphores. Nachos assumes that only a single user program exists at a given time. There are two versions of the Nachos filesystem. A stub version is simply a front-end to the Unix filesystem, so that users can access files within Nachos without having to write their own file system. The second version allows students to implement a filesystem on top of a raw disk, which is in fact a regular Unix file that can only be accessed through a simulated disk. The simulated disk accesses the underlying Unix file exclusively through operations that read and write individual sectors. Both file systems provide the same service and interface.\\

Similar to Nachos, Topsy, or Teachable Operating System, is an instructional tool developed by ETH-Zurich. In Topsy all threads are running in one address space and may share global memory between them. Synchronization of shared memory is accomplished via messages (AOSC, section 4.5.1AOSC) between multiple threads. The private area of a thread is its stack which is not protected against (faulty) accesses from other threads. Topsy divides the memory into two address spaces: one for user threads and the other for the OS kernel (separation of user and kernel process). , the two address spaces are embedded in one virtual address space, although no swapping of pages to secondary memory is supported. The memory is organized in paged manner. It provides a framework for writing and installing hardware device drivers. The kernel contains three main modules reflecting the basic OS tasks: the memory manager, the thread manager and the I/O subsystem.\\

Almost every other instructional operating system provides a skeleton of an operating system. However in eXpOS, only the specification has
been laid out, and students learn to implement XOS from ground up using the tools provided. In this project, a simple high level language called ExpL (Experimental Language) and its cross-compiler to XSM instruction set is provided to write user programs to test eXpOS. An XSM
dependent language called SPL (System Programmers Language) and its cross-compiler is provided to program the OS itself. Most instructional operating systems use the UNIX filesystem for file management by the operating system. Instead, eXpOS provides a native file system known as eXpFS (Experimental File System).


