\chapter{Hardware Interrupts and Exception Handlers}
\label{chap6}

When the machine is powered on, the bootstrap loader loads the first block of the disk into a pre-defined area in memory and transfers control to the newly loaded code called the OS startup code. The OS startup code loads the system call routines into memory. In addition to these, the timer interrupt handler, the exception handler, the disk interrupt handler and the terminal handler are also loaded. If the architecture supports other devices, then the corresponding device interrupt handlers also must be loaded by the OS Startup code. 

\section{Timer interrupt handler}
The hardware requirement specification for eXpOS assumes that the machine is equipped with a timer device that sends periodic hardware interrupts. The OS scheduler is invoked by the hardware timer interrupt handler. eXpOS specification suggest that a co-operative multitasking round robin scheduling is employed. This means that a round robin scheduling is employed, but a process may go to sleep inside a system call when:
\begin{itemize}
\item The resource which the process is trying to access (like a file or semaphore) is locked by another process (or even internally locked by another OS system call in concurrent execution). 
\item There is a disk or I/O device access in a system call which is slow. If the wait for the device access is to be avoided, there must be hardware support from the device to send a hardware interrupt when device operation is finished. This allows the OS to put the process on sleep for now, continue scheduling the remaining processes in round robin fashion and then wake up the sleeping process when the device sends the interrupt. 
\end{itemize}

\section{Exception handler}
If a process generates an illegal instruction, an invalid address (outside its virtual address space) or do a division by zero (or other faulty conditions which are machine dependent), the machine will generate an exception. The exception handler must terminate the process, wake up all processes waiting for it (or resources locked by it) and invoke the scheduler to continue round robin scheduling the remaining processes. The exception handler is also invoked when a page fault occurs. The module which handles demand paging (if the machine hardware supports demand paging) is invoked by the exception handler when there is a page fault. 

\section{Disk Controller}
eXpOS treats the disk as a special block device and assumes that the hardware provides low level block transfer routines to transfer disk blocks to memory (pages) and back. The block transfer routines contain instructions to initiate block-memory transfer by the disk controller hardware. After initiating the disk-memory transfer, the block transfer routine normally returns to the calling program, which sleeps for the disk operation to complete. When the disk-memory transfer is complete, the disk controller raises a hardware interrupt. The disk interrupt handler is responsible for waking processes that went into sleep awaiting completion of the disk operation.

\section{Terminal and other Device Handlers}
All other data handling devices (other than the disk) are treated as stream devices. It is assumed that for each device there are associated low level routines that can be invoked by the OS to transfer data and control instructions. Some of these devices may raise a hardware interrupt when the transfer is complete. Thus, for each device that raises an interrupt, there must be a corresponding device interrupt handler. 

The standard input and the standard output are two special stream devices with predefined identifiers STDIN = -1 and STDOUT = -2. Standard output permits only Write and standard input permits only Read. The Read operation typically puts the process executing the operation to sleep for console input from the user. When the user inputs data, the console device must send a hardware interrupt. The corresponding handler routine is called terminal handler. The terminal handler is responsible for waking up processes that are blocked for input console.
