\chapter{Algorithms Used}
\label{chap9}
%\section{File System calls}
\renewcommand{\algorithmicrequire}{\textbf{Input:}}
\renewcommand{\algorithmicensure}{\textbf{Output:}}

\begin{algorithm}
\caption{Create System Call}
\begin{algorithmic}
\REQUIRE Filename
\ENSURE 0 (Success) or -1 (No Space for file)
\STATE If the file is present in the system, return 0.   
\STATE Find the index of a free entry in the Inode Table. If no free entry found, return -1.  
\STATE Allocate the Inode Table entry to the file.
\STATE Update the Root file by adding an entry for the new file.  
\RETURN 0 
\end{algorithmic}
\end{algorithm}


\begin{algorithm}
\caption{Delete System Call}
\begin{algorithmic}
\REQUIRE Filename
\ENSURE 0 (Success) or -1 (File not found) or -2 (File is open)
\STATE If file is not present in Inode Table or if it is not a DATA file, return -1.
\STATE Find the index of the file in the Inode Table.
\STATE If File Table entry exists with the same index as found above, return -2. 
\STATE Update Inode Table.
\STATE Update the Root file by invalidating the root entry for the file.
\RETURN 0 
\end{algorithmic}
\end{algorithm}

%\subsection{Open System Call}

\begin{algorithm}
\caption{Open System Call}
\begin{algorithmic}
\REQUIRE Filename
\ENSURE File Descriptor (Success) or -1 (File not found) or -2 (Process has reached its limit of resources) or -3 (System has reached its limit of open files)
\STATE If file is not present in Inode Table or if it is of type EXEC, return -1.
\STATE Find the index of the Inode Table entry of the file.
\STATE Allocate entry in Per Process Resource Table
\IF{file is already open}
    \STATE Get the File Table entry of the file and increment the File Open Count.
\ELSE
    \STATE Find a free entry in the File Table. If there are no free entries, return -3.
\ENDIF
\STATE Allocate the Per-Process Resource Table entry to the file.
\STATE Update the File Table entry.
\RETURN Index of the Per-Process Resource Table entry.
\end{algorithmic}
\end{algorithm}

%\subsection{Close System Call}

\begin{algorithm}
\caption{Close System Call}
\begin{algorithmic}
\REQUIRE File Descriptor
\ENSURE 0 (Success) or -1 (File Descriptor is invalid) or -2 (File is locked by calling process)
\STATE If file descriptor does not correspond to valid entry in Per Process Resource Table, return -1   
\STATE Get the index of the File Table entry from Per-Process Resource Table entry.
\STATE If file is locked by the current process, return -2.   
\STATE In the File Table Entry found above, decrement the File Open Count. If the count becomes 0, invalidate the entry.  
\STATE Invalidate the Per-Process Resource Table entry.    
\RETURN 0
\end{algorithmic}
\end{algorithm}

%\subsection{Read System Call}

\begin{algorithm}
\caption{Read System Call}
\begin{algorithmic}
\REQUIRE File Descriptor and a Buffer (a String/Integer variable) into which a word is to be read from the file
\ENSURE 0 (Success) or -1 (File Descriptor is invalid) or -2 (File pointer has reached the end of file)
\IF{input is to be read from terminal}
    \STATE Wait till terminal gets the input for the current process.
    \STATE Copy the word from the input buffer of the Process Table to the buffer passed as argument and return with 0. 
\ENDIF
\STATE If file descriptor does not correspond to a valid entry in Per Process Resource Table, return -1.
\STATE Check the validity of the File pointer
\STATE Find the block and the position in the block from which input is read.
\WHILE{the file is locked by a process other than the current process}
    \STATE put the process to sleep and call the scheduler.
\ENDWHILE
\STATE Lock the file.
\STATE Get the buffer page number from block number.
\WHILE{the buffer is locked by a process other than the current process}
    \STATE put the process to sleep and call the scheduler.
\ENDWHILE 
\STATE Lock the buffer page.
\IF{the buffer contains a block other than the required block}
    \STATE Bring the block to the buffer from the disk
\ENDIF 
\STATE Copy the word at the offset position of the block into the buffer passed as argument and Increment lseek position.
\STATE Unlock the buffer and wake up all processes waiting for the buffer.
\STATE Unlock the file and wake up all processes waiting for the file.
\RETURN 0
\end{algorithmic}
\end{algorithm}

%\subsection{Write System Call}

\begin{algorithm}
\caption{Write System Call}
\begin{algorithmic}
\REQUIRE File Descriptor and a Word to be written
\ENSURE 0 (Success) or -1 (File Descriptor given is invalid) or -2 (No disk space)
\IF{word is to be written to standard output}
    \STATE Issue the machine instruction to output the given word and return 0.
\ENDIF
\STATE If file descriptor does not correspond to valid entry in Per Process Resource Table, return -1.
\STATE Get the index of the File Table entry and lseek position from the Per Process Resource Table entry.
\STATE If lseek is equal to MAX\_FILE\_SIZE - 1, return -2.  
\STATE Find the block and position in the block to which data has to be written.
\WHILE{the file is locked by a process other than the current process}
    \STATE put the process to sleep and call the scheduler.
\ENDWHILE
\STATE Lock the file.
\STATE Get the buffer page number from block number.
\WHILE{the buffer is locked by a process other than the current process}
    \STATE put the process to sleep and call the scheduler.
\ENDWHILE 
\STATE Lock the buffer page.
\IF{the buffer contains a block other than the required block}
    \STATE Bring the block to the buffer from the disk
\ENDIF
\STATE Copy the word passed as argument to the offset position in the buffer.  
\STATE Set dirty bit to 1 in the Buffer Table entry and increment lseek position.
\STATE Increment file size in the Inode Table entry and Root file entry.
\STATE Unlock the buffer and wake up all processes waiting for buffer.
\STATE Unlock the file and wake up all processes waiting for the file.
\RETURN 0.   
\end{algorithmic}
\end{algorithm}

%\subsection{Seek System Call}

\begin{algorithm}
\caption{Seek System Call}
\begin{algorithmic}
\REQUIRE File Descriptor and Offset
\ENSURE 0 (Success) or -1 (File Descriptor given is invalid) or -2 (Offset value moves the file pointer to a position outside the file)
\STATE If file descriptor does not correspond to valid entry in Per Process Resource Table, return -1.
\STATE Get the index of the File Table entry and lseek position from the Per Process Resource Table entry.
\STATE Check the validity of the given offset
\IF{given offset is 0}
    \STATE Set lseek value in the Per-Process Resource Table entry to 0.
\ELSE
    \STATE Change the lseek value in the Per-Process Resource Table entry to lseek+offset.
\ENDIF
\RETURN 0
\end{algorithmic}
\end{algorithm}

%\section{Process System calls}

%\subsection{Fork System Call}

\begin{algorithm}
\caption{Fork System Call}
\begin{algorithmic}
\REQUIRE None
\ENSURE Process Identifier to the parent process and 0 to child process (Success) or -1 (Number of processes has reached its maximum, returned to parent)
\STATE If no free entry in the Process Table, return -1.  
\STATE Find the index of a free entry in the Process Table and set the PID and PPID field.   
\STATE Count the number of valid stack pages from the Page Table of the parent process.   
\STATE If sufficient number of free pages are not present in memory, then increment the WAIT\_MEM\_COUNT in the System Status Table.
\WHILE{sufficient number of free pages are not present in memory}
    \STATE put the process to sleep and call the scheduler.
\ENDWHILE
\FOR{each stack page of the parent process}
    \IF{the stack page is valid}
        \STATE Allocate a free page to the child.
        \STATE Copy the stack page of the parent into the child stack page.
        \ELSE 
            \STATE Share the swap block containing the stack page with the child.
    \ENDIF
\ENDFOR
\STATE Construct the context of the child process.
\STATE Set the return value to 0 for the child process.
\RETURN The PID of the child process to the parent process.

\end{algorithmic}
\end{algorithm}

%\subsection{Exec System Call}

\begin{algorithm}
\caption{Exec System Call}
\begin{algorithmic}
\REQUIRE Filename
\ENSURE -1 (File not found or file is of invalid type)
\STATE If file not found in system or file type is not EXEC, return -1. \STATE For each page of the current process that is swapped out, find the swap block and decrement its entry in the Disk Free List. 
\STATE Setup code and library pages.
\STATE Free the heap pages
\STATE In the Process Table entry of the current process, set the Inode Index field to the index of Inode Table entry for the file.
\STATE Close all files opened by the current process.     
\STATE Release all semaphores held by the current process.   
\STATE Set SP to the start of the stack region and IP to the start of the code region. 
\RETURN
\end{algorithmic}
\end{algorithm}

%\subsection{Exit System Call}

\begin{algorithm}
\caption{Exit System Call}
\begin{algorithmic}
\REQUIRE None
\ENSURE None
\STATE If no more processes to schedule, shutdown the machine.    
\STATE Unlock and close all files opened by the current process.   
\STATE Release all the semaphores used by the current process.     
\STATE Wake up all processes waiting for the current process.     
\STATE Invalidate the Process Table entry and the page table entries of the current process
\STATE Invoke the scheduler to schedule the next process.  
\end{algorithmic}
\end{algorithm}

%\subsection{Getpid System Call}

\begin{algorithm}
\caption{Getpid System Call}
\begin{algorithmic}
\REQUIRE None
\ENSURE Process Identifier (Success) 
\STATE Find the PID of the current process from the Process Table.
\RETURN the PID of current process. 
\end{algorithmic}
\end{algorithm}

%\subsection{Getppid System Call}

\begin{algorithm}
\caption{Getppid System Call}
\begin{algorithmic}
\REQUIRE None
\ENSURE Process Identifier (Success) 
 \STATE Find the PPID of the current process from the Process Table.
\RETURN PPID.
\end{algorithmic}
\end{algorithm}

%\section{System Calls for access control and synchronization}

%\subsection{Wait System Call}

\begin{algorithm}
\caption{Wait System Call}
\begin{algorithmic}
\REQUIRE Process Identifier of the process for which the current process has to wait.
\ENSURE 0 (Success) or -1 (Given process identifier is invalid or it is the pid of the invoking process)
\STATE If process is intending to wait for itself or for a non-existent process, return -1.    
\STATE Change the status from (RUNNING, \_ ) to (WAIT\_PROCESS, Argument\_PID) in the Process Table.
\STATE Invoke the Scheduler to schedule the next process.
\RETURN 0
\end{algorithmic}
\end{algorithm}

%\subsection{Signal System Call}

\begin{algorithm}
\caption{Signal System Call}
\begin{algorithmic}
\REQUIRE None
\ENSURE 0 (Success) 
\STATE Wake up all processes waiting for the current process.
\RETURN 0
\end{algorithmic}
\end{algorithm}

%\subsection{FLock System Call}

\begin{algorithm}
\caption{FLock System Call}
\begin{algorithmic}
\REQUIRE File Descriptor
\ENSURE 0 (Success) or -1 (File Descriptor is invalid)
\STATE If file descriptor does not correspond to valid entry in Per Process Resource Table, return -1.
\WHILE{the file is locked by a process other than the current process}
    \STATE Change the state to (WAIT\_FILE, ftindex) where ftindex is the File Table index of the locked file and call the Scheduler.
\ENDWHILE
\STATE Change the ULock field of the file table to PID of current process.     
\RETURN 0
\end{algorithmic}
\end{algorithm}

%\subsection{FUnLock System Call}

\begin{algorithm}
\caption{FUnLock System Call}
\begin{algorithmic}
\REQUIRE File Descriptor
\ENSURE 0 (Success) or -1 (File Descriptor is invalid) or -2 (File was not locked by the calling process)
\STATE If file descriptor does not correspond to a valid entry in Per Process Resource Table, return -1.
\IF{file is locked}
    \STATE If current process has not locked the file, return -2.
    \STATE Unlock the file.
    \STATE Wake up all processes waiting for the file.
\ENDIF
\RETURN 0
\end{algorithmic}
\end{algorithm}


%\subsection{Semget System Call}

\begin{algorithm}
\caption{Semget System Call}
\begin{algorithmic}
\REQUIRE None
\ENSURE semaphore descriptor (Success) or -1 (Process has reached its limit of resources) or -2 (Number of semaphores has reached its maximum)
\STATE Find the index of a free entry in Per Process Resource Table. If no free entry, then return -1.
\STATE Find the index of a free entry in Semaphore table and increment the process count. If no free entry, return -2.
\STATE Store the index of the Semaphore table entry in the Per Process Resource Table entry.  
\RETURN Semaphore Table entry index.  
\end{algorithmic}
\end{algorithm}

%\subsection{Semrelease System Call}

\begin{algorithm}
\caption{Semrelease System Call}
\begin{algorithmic}
\REQUIRE Semaphore Descriptor
\ENSURE 0 (Success) or -1 (Semaphore Descriptor is invalid)
\STATE If Semaphore descriptor is not valid or the entry in Per Process Resource Table is not valid , return -1.
\IF{semaphore is locked by the current process}
    \STATE Unlock the semaphore before release.
\ENDIF
\STATE Decrement the process count of the semaphore.
\STATE Invalidate the Per-Process resource table entry.  
\RETURN 0
\end{algorithmic}
\end{algorithm}

%\subsection{SemLock System Call}

\begin{algorithm}
\caption{SemLock System Call}
\begin{algorithmic}
\REQUIRE Semaphore Descriptor
\ENSURE 0 (Success or the semaphore is already locked by the current process) or -1 (Semaphore Descriptor is invalid)
\\
\STATE If Semaphore descriptor is not valid or the entry in Per Process Resource Table is not valid , return -1.
\WHILE{the semaphore is locked by a process other than the current process}
    \STATE Put the current process to sleep 
    \STATE Call scheduler
\ENDWHILE
\STATE Change the Locking PID to PID of the current process in the Semaphore Table.
\RETURN 0
\end{algorithmic}
\end{algorithm}

%\subsection{SemUnLock System Call}

\begin{algorithm}
\caption{SemUnLock System Call}
\begin{algorithmic}
\REQUIRE Semaphore Descriptor
\ENSURE  0 (Success) or -1 (Semaphore Descriptor is invalid) or -2 (Semaphore was not locked by the calling process)
\STATE If Semaphore descriptor not valid or the entry in Per Process Resource Table is not valid , return -1.
\IF{semaphore is locked}
    \STATE If current process has not locked the semaphore, return -2.
    \STATE Unlock the semaphore.
    \STATE Wake up all processes waiting for the semaphore.
\ENDIF
\RETURN 0
\end{algorithmic}
\end{algorithm}


%\section{Hardware Interrupts and Exception Handlers}

%\subsection{Timer Interrupt Handler}

\begin{algorithm}
\caption{Timer Interrupt Handler}
\begin{algorithmic}
\STATE Save the context of current process and mark it as ready.
\STATE Increment TICK   
\IF{free memory pages present}
    \IF{there are sleeping processes that requires memory pages}
        \STATE Wake up all processes that requires memory pages.
    \ELSE
        
        \IF{disk is free and there are swapped processes}
           
            \STATE Swap in the seniormost swapped out process.
           
        \ENDIF
    \ENDIF
\ELSE
   \IF{disk is free and there are processes that require memory pages}
      
            \STATE Use the modified second chance algorithm to find an unreferenced page 
            \IF{the unreferenced page is a code page}
                \STATE In the page table entry of the unreferenced page, store the code block number.
            \ELSE
                \STATE If the unreferenced page is a stack or heap page, swap the page to a free swap block in the disk.
            \ENDIF
            \IF{unreferenced page is pointed to by the stack pointer of the process}
                \STATE Mark the process that owns the page as swapped out.
            \ENDIF
        
   \ENDIF
  
\ENDIF
\STATE Schedule the next ready process


\end{algorithmic}
\end{algorithm}

%\subsection{Disk Interrupt Handler}

\begin{algorithm}
\caption{Disk Interrupt Handler}
\begin{algorithmic}
\STATE In the Disk Status Table, set the STATUS field to 0, indicating that the disk is no longer busy.
\IF{load/store was issued by a process}  
    \STATE Wake up all processes waiting for the disk.
    \STATE If the loaded block was a swap block, decrement the corresponding Disk Free List entry.
\ELSE
   
    \STATE Get the PID of the process for which the scheduler had issued the load/store instruction.
    \IF{the disk operation was a load operation }
        \STATE Update Disk Free List.
        \STATE Update the Page Table of the process.
        \STATE Mark the process as ready. 
        \STATE Decrement SWAPPED\_COUNT in System Status Table.
    \ELSE 
        \STATE Update Memory Free List.
        \STATE Update Page Table of the process.
        \STATE Increment the MEM\_FREE\_COUNT in the System Status Table.
    \ENDIF
\ENDIF
\end{algorithmic}
\end{algorithm}

%\subsection{Exception Handler}

\begin{algorithm}
\caption{Exception Handler}
\begin{algorithmic}
\STATE If the exception is not caused by a page fault, display the cause and exit the process.
\STATE Find the page table entry of the page causing the exception.
\STATE If there are no free pages in the memory, then increment the WAIT\_MEM\_COUNT in the System Status Table.
\WHILE{there are no free pages in memory}
    \STATE Put the current process to sleep and call scheduler
\ENDWHILE
\STATE Find a free page in memory.
\IF{the page that caused exception is stored in the disk}
    \WHILE{the disk is busy}
        \STATE Put the current process to sleep and call scheduler
    \ENDWHILE
    \STATE Load the block to the memory page found above.
    \STATE Put the current process to sleep and call scheduler.
\ENDIF
\STATE Update the Page Table entry of the page that caused exception. 
\RETURN
\end{algorithmic}
\end{algorithm}

%\subsection{Terminal Interrupt Handler}

\begin{algorithm}
\caption{Terminal Interrupt Handler}
\begin{algorithmic}
\STATE Set the status field in Terminal Status Table to 0.
\STATE Locate the Process Table entry of the process that read the data.
\STATE Copy the word read from standard input to the Input Buffer field in the Process Table entry.
\STATE Set the PID field of Terminal Status Table to -1.
\STATE Wake up all processes waiting for terminal.
\end{algorithmic}
\end{algorithm}


%\section{Miscellaneous}

%\subsection{OS Startup Code}

\begin{algorithm}
\caption{OS Startup Code}
\begin{algorithmic}
\STATE Load the software interrupts and handler routines to memory.
\STATE Load the init program to memory and set up its machine context.
\STATE Load the idle program to memory and set up its machine context.
\STATE Load the disk data structures to the memory.
\STATE Initialize all the memory data structures.
\STATE Schedule the init process for execution.
\end{algorithmic}
\end{algorithm}

%\subsection{Shutdown System Call}

\begin{algorithm}
\caption{Shutdown System Call}
\begin{algorithmic}
\WHILE{disk is not free}
    \STATE Put the current process to sleep and call scheduler
\ENDWHILE
\STATE Store Inode Table to the disk.
\STATE Store dirty pages contained in buffer to the disk.     
\STATE Store Disk Free List to the disk.
\STATE Halt the machine.
\end{algorithmic}
\end{algorithm}

%\subsection{Shell process}

\begin{algorithm}
\caption{Shell process}
\begin{algorithmic}
\WHILE{true}
    \STATE Get the program to be executed using read system call
    \STATE If the shutdown instruction is received, invoke shutdown system call.
    \STATE childPID = fork();
    \IF{childPID == 0}
        \STATE Execute the program given by read.
    \ELSE 
        \STATE Wait for child to finish execution.
    \ENDIF
\ENDWHILE

\end{algorithmic}
\end{algorithm}

%\subsection{Idle process}

\begin{algorithm}
\caption{Idle process}
\begin{algorithmic}
\WHILE{true}
\ENDWHILE
\end{algorithmic}
\end{algorithm}
