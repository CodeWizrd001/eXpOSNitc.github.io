\chapter{Experimental File System(eXpFS)}
\label{chap3}
\section{Introduction}
 
The eXpFS logical file system provides a file abstraction that allows application programs to think of each data (or executable) file stored in the disk as a continuous stream of data (or machine instructions). eXpOS provides a sequence of file system calls through which application programs can create/read/write files. These system calls are OS routines that does the translation of the user request into physical disk block operations.
\\

In addition to the eXpOS system call interface, the eXpFS specification also requires that there is an external interface through which executable and data files can be loaded into the file system externally. The external interface for eXpOS implementation on the XSM machine is the XFS Interface.
\\

\section{eXpFS File system organization}
The eXpFS logical file system comprises of files organized in a single directory called the root. The root is also treated conceptually as a file. Every eXpFS file is a sequence of words. Associated with each eXpFS file there are three attributes - name, size and type, each attribute being one word long. The filename must be a string. Each file must have a unique name. The size of the file will be the total number of words stored in the file. There are three types of eXpFS files - the root, data files and executable files. Each file in eXpFS has an entry in the root called its root entry. 

\subsection{The eXpFS Root File}
The root file has the name root and contains meta-data about the files stored in the file system. For each file, the root stores a root entry which consists of filename, file-size and file-type. The first root entry is for the root itself. The operations on the root file are Open, Close, Read and Seek. 

\subsection{eXpFS Data files}
A data file is a sequence of words. eXpFS treats every file other than root and executable files as a data file. The Create system call automatically sets the file type field in the root entry for any file created through the create system call to DATA. eXpOS allows an application program to perform the following operations on data files: Create, Delete, Open, Close, FLock, FUnlock, Read, Write, Seek.  

\subsection{eXpFS Executable files}
 These contain executable code for programs that can be loaded and run by the operating system. From the point of view of the eXpFS file system alone, executable files are just like data files except that file type is EXEC in the root entry. eXpFS specification does not allow executable files to be created by application programs. They can only be created externally and loaded using the external interface. However, application programs can read executable files.
\\

Executable files are essentially program files that must be loaded and run by the operating system. The Exec system call invokes the eXpOS loader that loads an executable file in XEXE format into memory for execution. The Operating system imposes certain structure on these files (called the executable file format). Moreover, the instructions must execute on the machine on which the OS is running. Thus, there is dependency on the hardware as well.
\\

The executable file format recognized by eXpOS is called the Experimental executable file (XEXE) format. In this format, an executable file is divided into two sections. The first section is called header and the second section called the code (or text) section. The code section contains the program instructions. The header section contains information like the size of the text and data segments in the file, the space to be allocated for stack and heap areas when the program is loaded for execution etc. This information is used by the OS loader to map the file into a virtual address space and create a process in memory for executing the program.


